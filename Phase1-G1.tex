% "pdf"

\documentclass[a4paper,12pt]{article}

\usepackage[french]{babel}
\usepackage[T1]{fontenc}
\usepackage[utf8x]{inputenc}

\usepackage[margin=2cm]{geometry}
\usepackage{pifont}

\usepackage[colorlinks=true,
            linkcolor=black]{hyperref}
\usepackage{titletoc}
\usepackage{titlesec}

\usepackage{soul}
% \usepackage{ulem}

\usepackage[lined,ruled]{algorithm2e}

\title{\Huge\textbf{Rapport de Phase 1 du Groupe 1}}
\author{Lemarchand Benoît \& Megna Anaël \& Shimi Adam}
\date{Jeudi, 7 Mai 2015}

\usepackage{fancyhdr}
\pagestyle{fancy}
\renewcommand{\headrulewidth}{0pt}
\lfoot{} \cfoot{} \rfoot{\thepage} \lhead{} \chead{} \rhead{}

\usepackage{amsmath}
\usepackage{amssymb}
\usepackage{amsthm}
\usepackage{mathtools}
\usepackage{mathrsfs}
\usepackage{stmaryrd}
\usepackage{esint}
\usepackage{esvect}
\usepackage{multirow}

\newtheorem*{remark}{Remarque}

\usepackage{listings}

\renewcommand{\phi}{\varphi}
\renewcommand{\epsilon}{\varepsilon}

\setlength{\parindent}{1em}

\begin{document}

\begin{titlepage}
  \maketitle
  \thispagestyle{empty}
  \tableofcontents
\end{titlepage}

\section{Reconstruction et Classification de la solution d'un modèle
atmosphérique simplifié}

\subsection{Détermination de $\alpha$}

Résoudre le problème de minimisation en $\alpha$
$\displaystyle \min_{\alpha \in \mathbb{R}^{n_d}} \|U_{|t=0}.\alpha - Z_0\|_2$
revient à trouver $\alpha \in \mathbb{R}^{n_d}$ tel que
$\displaystyle U_{|t=0}^T.U_{|t=0}.\alpha = U_{|t=0}^T.Z_0$ (équations
normales).
Or $U_{|t=0}^T.U_{|t=0}$ est, de manière évidente, une matrice symétrique
positive et, de plus, par construction de $U$, ses valeurs propres sont non
nulles d'où son caractère défini. On décide donc de résoudre ce système par la
méthode de la \og steepest descent\fg, d'où l'algorithme 1.


\begin{algorithm}[H]
    $\epsilon = value$\;
    $\alpha = vector$\;
    $U'_0 = U_0^T . U_0$\;
    $Z'_0 = U_0^T . Z_0$\;
    $r = Z'_{0} - U'_0 . \alpha$\;
    \While{ $\displaystyle \frac{\|r\|_2}{\|Z'_0\|_2}  > \epsilon$ }{
        $\displaystyle \lambda = \frac{r^T.r}{(U_0 . r)^T . (U_0 . r)}$\;
        $\alpha = \alpha + \lambda r$\;
        $r = Z'_0 - U'_0 . \alpha$\;
    }
    \caption{Détermination de $\alpha$}
\end{algorithm}


\section{Méthode dérivée de la puissance itérée, analyse et implémentation}

    \subsection{Avantages et inconvénients de la méthode}

    \begin{remark}[Complexité en temps du calcul matriciel]
    Ici, nous utilisons les routines de calcul matriciel de MATLAB, qui viennent de BLAS.\footnote{Voir la routine DGEMM, \url{http://en.wikipedia.org/wiki/Basic_Linear_Algebra_Subprograms}} Etant donné que ces routines, quoique optimisées pour s'adapter au matériel et réduire les temps de calcul, n'implémentent ni l'algorithme de Strassen en $O(n^{2.807})$ ni aucun des developpements récents dans le domaine qui vont jusqu'au $O(n^{2.373})$ de Virginia V. Williams, nous supposerons que la complexité en temps d'une multiplication matricielle de MATLAB est en $O(n^3)$ pour une matrice de $\mathbb{R}^{n*n}$ et $O(nmk)$ pour des matrices de $\mathbb{R}^{n*m}$ et $\mathbb{R}^{m*k}$.
    \end{remark}

    Ici, nous analysons l'algorithme de la puissance itérée adapté aux sous-espaces propres, à la fois en terme d'avantages et d'inconvenients. Evidemment, ces termes n'ont de sens qu'au niveau d'une comparaison. \\
    D'où le choix d'un algorithme dit "de base", à l'aune duquel nous pourrons analyser celui qui fait l'objet de cette section.
    Les particularités de cet algorithme "de base" sont : \\

        \begin{itemize}
            \item Il calcule toutes les valeurs singulières et tous les vecteurs singuliers.
            \item Il est basé, d'après la documentation de MATLAB\footnote{MATLAB utilise la routine svd de LIPACK, dont l'algorithme est expliqué dans \\    J. Demmel, W. Kahan \textit{Accurate Singular Values of Bidiagonal Matrices}, submitted to SIAM J.Sci.Stat.Comput., v.11, n.5, pp.873-912, 1990}, sur l'application successive de transformation de Householder et de Rotation de Givens, avant d'appliquer une methode itérative de calcul des valeurs et vecteurs singuliers pour une matrice bidiagonale.
            \item Sa complexité asymptotique en temps est, toujours d'après l'implémentation MATLAB, en $O(MaxIter*(n*m^2))$, où MaxIter est le nombre d'itérations maximal de la seconde partie de l'algorithme.
            \item Celle en espace, basée sur celles des différentes étapes de l'algorithme, est en $O(n*m)$.
        \end{itemize}
\bigskip
    Par comparaison, notre nouvel algorithme a pour caractéristiques : \\

        \begin{itemize}
            \item Il calcule uniquement les valeurs singulières et les vecteurs singuliers à gauche nécessaire d'après le pourcentage d'information demandé.
            \item En ce qui concerne la complexité en temps pour A $\in \mathbb{R}^{m*m}$, nous obtenons comme formule asymptotique $O(MaxIter*(n*m^2))$. En effet, toutes les matrices impliquées dans notre algorithme sont de taille n*m, excepté $A$. Nous en déduisons d'après la remarque sur la complexité que l'ensemble des calculs matriciels de notre algorithme est en $O(n*m^2)$. Ajoutons a cela la partie itérative, bornée par MaxIter, et nous obtenons cette complexité.
            \item Quand à la complexité en espace, la version naive de l'algorithme, qui consiste à calculer A avant toute chose, nous donne une complexité en $O(m^2)$, étant donné que c'est la matrice la plus volumineuse.
        \end{itemize}
\bigskip
\ \\
\ \\
    Le premier point de comparaison est celui du temps de calcul. Comme nous l'avons vu, asymptotiquement, les deux algorithmes se comportent similairement en terme de temps. Cependant, notre algorithme ne va pas calculer toutes les valeurs singulières, sauf dans de très rares cas. Il semble donc sensé de supposer que si les deux algorithmes se valent dans le pire des cas, le notre a un léger avantage sur svd lorsque \textit{PercentTrace} est relativement éloigné de 1. \\

    Ensuite vient la question de l'espace. Ici, clairement, notre algorithme a un désavantage lié au calcul explicite et au stockage de $A$, qui fait augmenter la complexité asymptotique en espace. Si l'on considère en plus que m dans le cas pratique que nous implémentons ici est de l'ordre de $10^5$, ce désavantage est écrasant. Nous verrons dans la sous-section suivante comment se passer du stockage de $A$. \\

    Enfin, en ce qui concerne les avantages moins significatifs de notre algorithme, il y a le fait que dans la majorité des cas, il stocke moins de valeurs singulières que svd. Et dans tous les cas, il stocke moins de vecteurs singuliers, étant donné qu'il ne calcule à aucun moment les vecteurs singuliers à droite. Nous pouvons aussi remarquer que notre algorithme sort directement un sous-espace "singulier", tandis qu'il faut reconstruire celui-çi avec la sortie de svd.

    \subsection{Implémentation de l'algorithme}

    Comme expliqué dans la sous-section précédente, l'utilisation naive de l'algorithme dérivé de la puissance itérée pour remplacer svd nécessite le calcul de $A = Z*Z^T$, dont l'ordre de grandeur du nombre de coefficients est de $10^{10}$. Ceci fait que le calcul même de $A$ entraine un overflow dans MATLAB, nous forçant à adapter l'algorithme pour l'utiliser. \\

    Voici les moments où nous avons dû modifier l'algorithme fourni pour éviter de calculer $A$ : \\

        \begin{itemize}
            \item Le calcul de $Y = A^p*V$. Ici, il suffit de remplacer $A$ par $Z*Z^T$, et d'effectuer les calculs. A aucun moment les matrices intermédiaires n'auront une taille supérieure à $n*m$, ce qui est assez petit pour ne pas causer d'overflow dans MATLAB.
            \item Le calcul du coefficient de Rayleigh $H = V^T*A*V$. Ici, pour gagner en espace, il suffit de décomposer $H = V^T*Z*Z^T*V$, d'où $H = (Z^T*V)^T*(Z^T*V)$. Nous en déduisons que calculer $Z^T*V$ puis d'une transposition et d'un produit matriciel pour calculer le coefficient de Rayleigh.
        \end{itemize}

%%%%%%

% exemple d'un algorithme

\begin{algorithm}[H]  % toujours mettre le H
  \DontPrintSemicolon % pour pas afficher ;
  initialization\;    % toujours terminer avec \;
  \BlankLine          % pour sauter une ligne
  \While{not at end of this document}{
    read current\;
    \eIf{understand}{ % eIf = if, else, end / If = if, end
      go to next section\;
      current section becomes this one\;
    }{
      go back to the beginning of current section\;
    }
  }
  \caption{How to write algorithms} % apparait dans la liste des algorithmes
\end{algorithm}

\newpage
\section*{Annexes}

\listofalgorithms

\end{document}
