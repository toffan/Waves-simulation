% "pdf"

\documentclass[a4paper,12pt]{article}

\usepackage[french]{babel}
\usepackage[T1]{fontenc}
\usepackage[utf8x]{inputenc}

\usepackage[margin=2cm]{geometry}
\usepackage{pifont}

\usepackage[colorlinks=true,
            linkcolor=black]{hyperref}
\usepackage{titletoc}
\usepackage{titlesec}

\usepackage{soul}
\usepackage{ulem}

\usepackage[lined,ruled]{algorithm2e}

\title{\Huge\textbf{Rapport de Phase 1 du Groupe 1}}
\author{Lemarchand Benoît \& Megna Anaël \& Shimi Adam}
\date{Jeudi, 7 Mai 2015}

\usepackage{fancyhdr}
\pagestyle{fancy}
\renewcommand{\headrulewidth}{0pt}
\lfoot{} \cfoot{} \rfoot{\thepage} \lhead{} \chead{} \rhead{}

\usepackage{amsmath}
\usepackage{amssymb}
\usepackage{amsthm}
\usepackage{mathtools}
\usepackage{mathrsfs}
\usepackage{stmaryrd}
\usepackage{esint}
\usepackage{esvect}
\usepackage{multirow}

\usepackage{listings}

\renewcommand{\phi}{\varphi}
\renewcommand{\epsilon}{\varepsilon}

\setlength{\parindent}{1em}

\begin{document}

\begin{titlepage}
  \maketitle
  \thispagestyle{empty}
  \tableofcontents
\end{titlepage}

\section{Reconstruction et Classification de la solution d'un modèle
atmosphérique simplifié}

\section{»» Donner un titre à cette section ««} % TODO

%%%%%%

% exemple d'un algorithme

\begin{algorithm}[H]  % toujours mettre le H
  \DontPrintSemicolon % pour pas afficher ;
  initialization\;    % toujours terminer avec \;
  \BlankLine          % pour sauter une ligne
  \While{not at end of this document}{
    read current\;
    \eIf{understand}{ % eIf = if, else, end / If = if, end
      go to next section\;
      current section becomes this one\;
    }{
      go back to the beginning of current section\;
    }
  }
  \caption{How to write algorithms} % apparait dans la liste des algorithmes
\end{algorithm}

\newpage
\section*{Annexes}

\listofalgorithms

\end{document}
